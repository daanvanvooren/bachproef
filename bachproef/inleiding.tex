%%=============================================================================
%% Inleiding
%%=============================================================================

\chapter{\IfLanguageName{dutch}{Inleiding}{Introduction}}
\label{ch:inleiding}

\begin{displayquote}
	“It takes 20 years to build a reputation and few minutes of cyber-incident to ruin it.” – Stephane Nappo
\end{displayquote}

De quote van Stephane Nappo, Global Chief Information Security Officer, expliceert met zijn quote het belang van cybersecurity in een notendop. Cybersecurity is namelijk één van de meest besproken technologische onderwerpen die centraal staan in bedrijven die 'oorlog' voeren met cyber criminelen. Information and Communications Technology wordt de dag vandaag alsmaar grootschaliger toegepast en zal in de nabije toekomst zeker niet verminderen. Door de grote uitbreiding in Information and Communications Technology stijgt de kans op cyber aanvallen evenredig mee. Persoonlijke gegevens of resources kunnen door de stijging in cyber aanvallen sneller gestolen worden dan 20 jaar geleden. Een oplossing was dus hoognodig, waardoor het cybersecurity begrip ontstond. 
\newline
\newline
Daarbij moeten organisaties sterk investeren om de aanvalspogingen drastisch te doen verminderen om zo de cyber criminelen een stapje voor te zijn. Nieuwe principes zoals 'Internet of Things' en 'Bring Your Own Device' maken bedrijven het er niet gemakkelijker op. Tegenwoordig worden 'Internet of Things' en 'Bring Your Own Device' principes bijna in alle bedrijven toegepast. Bedrijven moeten met als gevolg de proliferatie netwerk-compatibele apparaten beter ondersteunen, zelfs wanneer een groot aantal cyber bedreigingen en veel gepubliceerde datalekken aantonen dat het belang van netwerken bescherming van cruciaal belang is. 
\newline
\newline
Door de 'Internet of Things' en 'Bring Your Own Device' principes 
Vanuit dit probleem kwam Axians met de vraag voor dit onderzoek, waarbij het gebruik van Cisco Identity Services Engine in samen werking met 'Policy-based access control', 'Port-based access control' en Thread-Centric access control centraal staat. Cisco Identity Services Engine is een network access control product dat de handhaving van beveiligings- en toegangsbeleid mogelijk maakt voor eind apparaten die zijn aangesloten op het netwerk apparatuur van het bedrijf. Het doel is om identiteitsbeheer en om het veiligsbeheer op verschillende eind apparaten en applicaties te vereenvoudigen.
\newline
\newline
Hieruit is de onderzoeksvraag '\textit{Een bedrijfsnetwerk beter beveiligen en beheren met behulp van Cisco Identity Services Engine}' ontstaan, waarbij volgende doelstellingen mee bedacht zijn: 

\begin{itemize}
	\item Implementatie van Cisco Identity Services Engine in een netwerk.
	\item Integratie van 'Port-based access control met Cisco Identity Services Engine in een netwerk.
	\item Integratie van 'Policy-based access control met Cisco Identity Services Engine in een netwerk.
	\item Integratie van 'Thread-Centric access control met Cisco Identity Services Engine in een netwerk.
	\item Evaluatie van Cisco Identity Services Engine in een netwerk.
\end{itemize}

De samenleving en de Information and Communications technologieën zullen steeds verder evolueren waardoor ondernemingen in staat moeten om cybersecurity steeds te blijven opvolgen. Men zal constant moeten bijleren om de strijd tegen cyberaanvallen te blijven winnen. De opvolging wordt er zeker en vast niet gemakkelijker op, waardoor dit efficiënter verloopt als men in team werkt. Cybersecurity vereist dus een samenwerking om de resources en gegevens te beschermen.

\section{\IfLanguageName{dutch}{Probleemstelling}{Problem Statement}}
\label{sec:probleemstelling}
Aangezien Information and Communications Technology zou centraal staat in de 21ste eeuw kan iedereen in aanraking komen met cyber aanvallen. Denk maar aan de talrijke pogingen die cyber criminelen via Facebook, Twitter, E-mail, enz. proberen om accounts, geld, informatie te bemachtigen bij deze slachtoffers. Veel van deze gevallen gebeuren op individueel vlak, maar bedrijven hebben evenveel risco's om besloten te worden. Men kan verschillende methodes bedenken hoe organisaties het slachtoffer kan zijn van cyberaanvallen. Maar dit onderzoek focust zich pricipes zoals 'Internet of Things' en 'Bring Your Own Device'.
\newline
\newline
Hierdoor wordt het doelgroep van dit onderzoek beperkt met oog op de eind apparaten die het netwerk steeds verlaten en opnieuw binnen komen, specifiek zal het onderzoek zich richten op de eind apparaten van de werknemers van Axians die gebruiken maken van het netwerk. Aangezien deze eind apparaten het netwerk flink kunnen toetakelen.


\section{\IfLanguageName{dutch}{Onderzoeksvraag}{Research question}}
\label{sec:onderzoeksvraag}
Een veiliger en beter beheerbaarder netwerk is wat men wil bereiken door het gebruik van een network access control product en use cases zoals :'\textit{Port-based access control, Policy-based access control}' en '\textit{Thread-Centric access control}'. Dit network access control product is Cisco Identity Services Engine.

De hoofdstuk onderzoeksvraag in deze bachelorproef is: 
\begin{displayquote}
	Hoe kunnen we een evoluerend bedrijfsnetwerk beter beveiligen door het gebruik van Cisco ISE?
\newline
\newline
\end{displayquote}
Verder zal deze bachelorproef zich verdiepen in het gebruik van Cisco's Identity Services Engine use cases. 
\begin{displayquote}
	Welke invloed heeft de integratie van \textit{Port-based access control} in een netwerk?  
\newline
\newline
	Wat is meerwaarde van de integratie van \textit{Policy-based access control} in een netwerk?  
\newline
\newline
	Kan het netwerk wel degelijk beschermt worden tegen verder uitbreiding van Malware door integratie van \textit{Thread-Centric access control} in een netwerk?  
\end{displayquote}

\section{\IfLanguageName{dutch}{Onderzoeksdoelstelling}{Research objective}}
\label{sec:onderzoeksdoelstelling}

%Wat is het beoogde resultaat van je bachelorproef? Wat zijn de criteria voor succes? Beschrijf die zo concreet mogelijk. Gaat het bv. om een proof-of-concept, een prototype, een verslag met aanbevelingen, een vergelijkende studie, enz.
Door middel van Cisco Identity Services Engine network access control product zal dit onderzoek inzicht bieden op het beveiligen en beter beheren van netwerken tegen interne of externe gevaren. Vervolgens moeten de resultaten van de enquête een eenduidig antwoord bieden die in lijn liggen met de resultaten van de uitgevoerde testen, enkel dan is dit deelonderzoek geslaagd. Door dit onderzoek wil men dat ondernemingen stappen zetten in de juiste richting tegen de strijd met cyber criminelen. Zo kan informatie, geld en andere resource in bedrijven gewaarborgd blijven. Wanneer bedrijven de waarde van een network access control inzien, dan is dit onderzoek volledig geslaagd.

\section{\IfLanguageName{dutch}{Opzet van deze bachelorproef}{Structure of this bachelor thesis}}
\label{sec:opzet-bachelorproef}

% Het is gebruikelijk aan het einde van de inleiding een overzicht te
% geven van de opbouw van de rest van de tekst. Deze sectie bevat al een aanzet
% die je kan aanvullen/aanpassen in functie van je eigen tekst.

De rest van deze bachelorproef is als volgt opgebouwd:

In Hoofdstuk~\ref{ch:stand-van-zaken} wordt een overzicht gegeven van de stand van zaken binnen het onderzoeksdomein, op basis van een literatuurstudie.

In Hoofdstuk~\ref{ch:methodologie} wordt de methodologie toegelicht en worden de gebruikte onderzoekstechnieken besproken om een antwoord te kunnen formuleren op de onderzoeksvragen.

In Hoofdstuk~\ref{ch:Proof of concept} wordt een overzicht gegeven van de omgeving en de gebruikte tools voor implemenatie van Cisco Identity Services Engine. 

In Hoofdstuk~\ref{ch:Resultaten} worden de resultaten van de uitgevoerde testen en van de enquête toegelicht.

In Hoofdstuk~\ref{ch:conclusie}, tenslotte, wordt de conclusie gegeven en een antwoord geformuleerd op de onderzoeksvragen. Daarbij wordt ook een aanzet gegeven voor toekomstig onderzoek binnen dit domein.